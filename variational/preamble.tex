% Input configuration
\usepackage[utf8]{inputenc}

% -- support vietnamese --
\usepackage[vietnamese]{babel}
% -- support english --
% \usepackage[english]{babel}
% ----------------------------------------------

% For math
\usepackage{amsmath,amssymb}
\usepackage{mathtools}
\usepackage{amsthm}
\newcommand{\N}{\mathbb{N}}
\newcommand{\Z}{\mathbb{Z}}
\newcommand{\R}{\mathbb{R}}
\newcommand{\F}{\mathbb{F}}
% ----------------------------------------------

% For images, colors
\usepackage{graphicx}
\usepackage{color,xcolor}
% ----------------------------------------------

% For captioning
\usepackage{caption}
\setbeamertemplate{caption}[numbered]
% ----------------------------------------------

% For tables
\usepackage{booktabs}
% ----------------------------------------------

% For hyper references between sections
\usepackage{hyperref}
% ----------------------------------------------

% For texts, algorithm and codes
\usepackage{ulem}
\usepackage{algorithm}
\usepackage{listings}
% ----------------------------------------------

% For tikz
\usepackage{framed}
\usepackage{tikz}
\usepackage{pgf}
\usetikzlibrary{calc,trees,positioning,arrows,chains,shapes.geometric,%
decorations.pathreplacing,decorations.pathmorphing,shapes,%
matrix,shapes.symbols}
\pgfmathsetseed{1} % To have predictable results
% Define a background layer, in which the parchment shape is drawn
\definecolor{doge}{HTML}{0065BD}
\pgfdeclarelayer{background}
\pgfsetlayers{background, main}

% Macro to draw the shape behind the text, when it fits completly in the page
% define styles for the normal border and the torn border
\tikzset{
  normal border/.style={doge!40!black!10, decorate, 
     decoration={random steps, segment length=2.5cm, amplitude=.7mm}},
  torn border/.style={doge!40!black!5, decorate, 
     decoration={random steps, segment length=.5cm, amplitude=1.7mm}}}

% Macro to draw the shape behind the text, when it fits completly in the
% page
\def\parchmentframe#1{
\tikz{
  \node[inner sep=2em] (A) {#1};  % Draw the text of the node
  \begin{pgfonlayer}{background}  % Draw the shape behind
  \fill[normal border] 
        (A.south east) -- (A.south west) -- 
        (A.north west) -- (A.north east) -- cycle;
  \end{pgfonlayer}}}

% Macro to draw the shape, when the text will continue in next page
\def\parchmentframetop#1{
\tikz{
  \node[inner sep=2em] (A) {#1};    % Draw the text of the node
  \begin{pgfonlayer}{background}    
  \fill[normal border]              % Draw the ``complete shape'' behind
        (A.south east) -- (A.south west) -- 
        (A.north west) -- (A.north east) -- cycle;
  \fill[torn border]                % Add the torn lower border
        ($(A.south east)-(0,.2)$) -- ($(A.south west)-(0,.2)$) -- 
        ($(A.south west)+(0,.2)$) -- ($(A.south east)+(0,.2)$) -- cycle;
  \end{pgfonlayer}}}

% Macro to draw the shape, when the text continues from previous page
\def\parchmentframebottom#1{
\tikz{
  \node[inner sep=2em] (A) {#1};   % Draw the text of the node
  \begin{pgfonlayer}{background}   
  \fill[normal border]             % Draw the ``complete shape'' behind
        (A.south east) -- (A.south west) -- 
        (A.north west) -- (A.north east) -- cycle;
  \fill[torn border]               % Add the torn upper border
        ($(A.north east)-(0,.2)$) -- ($(A.north west)-(0,.2)$) -- 
        ($(A.north west)+(0,.2)$) -- ($(A.north east)+(0,.2)$) -- cycle;
  \end{pgfonlayer}}}

% Macro to draw the shape, when both the text continues from previous page
% and it will continue in next page
\def\parchmentframemiddle#1{
\tikz{
  \node[inner sep=2em] (A) {#1};   % Draw the text of the node
  \begin{pgfonlayer}{background}   
  \fill[normal border]             % Draw the ``complete shape'' behind
        (A.south east) -- (A.south west) -- 
        (A.north west) -- (A.north east) -- cycle;
  \fill[torn border]               % Add the torn lower border
        ($(A.south east)-(0,.2)$) -- ($(A.south west)-(0,.2)$) -- 
        ($(A.south west)+(0,.2)$) -- ($(A.south east)+(0,.2)$) -- cycle;
  \fill[torn border]               % Add the torn upper border
        ($(A.north east)-(0,.2)$) -- ($(A.north west)-(0,.2)$) -- 
        ($(A.north west)+(0,.2)$) -- ($(A.north east)+(0,.2)$) -- cycle;
  \end{pgfonlayer}}}

\newenvironment{parchment}[1][Example]{%
  \def\FrameCommand{\parchmentframe}%
  \def\FirstFrameCommand{\parchmentframetop}%
  \def\LastFrameCommand{\parchmentframebottom}%
  \def\MidFrameCommand{\parchmentframemiddle}%
  \vskip\baselineskip
  \MakeFramed {\FrameRestore}
  \noindent\tikz\node[inner sep=1ex, draw=white,fill=doge!30, 
          anchor=west, overlay] at (0em, 2em) {\sffamily#1};\par}%
{\endMakeFramed}
% ----------------------------------------------

% Beamer theme configuration
\mode<presentation>{
    \usetheme{default}
    % Boadilla CambridgeUS
    % default Antibes Berlin Copenhagen
    % Madrid Montpelier Ilmenau Malmoe
    % Berkeley Singapore Warsaw
    % \usecolortheme{doge}
    % beetle, beaver, orchid, whale, dolphin
    \useoutertheme{infolines}
    % infolines miniframes shadow sidebar smoothbars smoothtree split tree
    \useinnertheme{circles}
    % circles, rectanges, rounded, inmargin
}

\setbeamertemplate{blocks}[rounded][shadow]
\setbeamercolor{block title}{bg=doge!40,fg=white}
\newcommand{\reditem}[1]{\setbeamercolor{item}{fg=red}\item #1}
\newcommand*{\Scale}[2][4]{\scalebox{#1}{\ensuremath{#2}}}
\renewcommand\textbullet{\ensuremath{\bullet}}
% ----------------------------------------------

% For flow chart and codes
% Flow chart settings
\tikzset{
    >=stealth',
    punktchain/.style={
        rectangle, 
        rounded corners, 
        % fill=black!10,
        draw=white, very thick,
        text width=6em,
        minimum height=2em, 
        text centered, 
        on chain
    },
    largepunktchain/.style={
        rectangle,
        rounded corners,
        draw=white, very thick,
        text width=10em,
        minimum height=2em,
        on chain
    },
    line/.style={draw, thick, <-},
    element/.style={
        tape,
        top color=white,
        bottom color=blue!50!black!60!,
        minimum width=6em,
        draw=blue!40!black!90, very thick,
        text width=6em, 
        minimum height=2em, 
        text centered, 
        on chain
    },
    every join/.style={->, thick,shorten >=1pt},
    decoration={brace},
    tuborg/.style={decorate},
    tubnode/.style={midway, right=2pt},
    font={\fontsize{10pt}{12}\selectfont},
}

% code settings
\lstset{
    language=C++,
    basicstyle=\ttfamily\footnotesize,
    keywordstyle=\color{red},
    breaklines=true,
    xleftmargin=2em,
    numbers=left,
    numberstyle=\color[RGB]{222,155,81},
    frame=leftline,
    tabsize=4,
    breakatwhitespace=false,
    showspaces=false,               
    showstringspaces=false,
    showtabs=false,
    morekeywords={Str, Num, List},
}
% ---------------------------------------------------------------------
% title page
% Sử dụng cho bảo vệ luận văn
% \makeatletter
% \newcommand\titlegraphicii[1]{\def\inserttitlegraphicii{#1}}
% \titlegraphicii{}
% \setbeamertemplate{title page}
% {
%     \vskip-0.5cm%
%     \begin{beamercolorbox}[sep=14pt,center]{institute}
%           \usebeamerfont{institute}\insertinstitute
%     \end{beamercolorbox}    
%    {\usebeamercolor[fg]{titlegraphic}\inserttitlegraphic\hfill\inserttitlegraphicii\par}
%   \begin{centering}
%     \begin{beamercolorbox}[sep=8pt,center]{title}
%       \usebeamerfont{title}\inserttitle\par%
%       \vskip0.5em
%       \ifx\insertsubtitle\@empty%
%       \else%
%         \vskip0.25em%
%         {\usebeamerfont{subtitle}\usebeamercolor[fg]{subtitle}\insertsubtitle\par}%
%       \fi%     
%         \end{beamercolorbox}%
%         \begin{beamercolorbox}[sep=8pt,center]{author}
%           \usebeamerfont{author}\insertauthor
%         \end{beamercolorbox}\vskip0.5em
%         % \vskip1em\par
%         \begin{beamercolorbox}[sep=8pt,center]{date}
%           \usebeamerfont{date}\insertdate
%         \end{beamercolorbox}%
%       \end{centering}
%   %\vfill
% }
% Sử dụng để trình bày
\makeatletter
\newcommand\titlegraphicii[1]{\def\inserttitlegraphicii{#1}}
\titlegraphicii{}
\setbeamertemplate{title page}
{
    \vskip-0.5cm%
   {\usebeamercolor[fg]{titlegraphic}\inserttitlegraphic\hfill\inserttitlegraphicii\par}
  \begin{centering}
    \begin{beamercolorbox}[sep=8pt,center]{title}
      \usebeamerfont{title}\inserttitle\par%
      \vskip0.5em
      \ifx\insertsubtitle\@empty%
      \else%
        \vskip0.25em%
        {\usebeamerfont{subtitle}\usebeamercolor[fg]{subtitle}\insertsubtitle\par}%
      \fi%     
        \end{beamercolorbox}%
        \begin{beamercolorbox}[sep=8pt,center]{author}
          \usebeamerfont{author}\insertauthor
        \end{beamercolorbox}\vskip0.5em
        \begin{beamercolorbox}[sep=14pt,center]{institute}
          \usebeamerfont{institute}\insertinstitute
        \end{beamercolorbox}    
        \vskip1em\par
        \begin{beamercolorbox}[sep=8pt,center]{date}
          \usebeamerfont{date}\insertdate
        \end{beamercolorbox}%
      \end{centering}
  %\vfill
}

\makeatother
\author{Presented by Author}
\title{Presentation Title}
\subtitle{Presentation Subtile}
\institute{Department \\ University}
\date{\today}

% frame title
% https://tex.stackexchange.com/questions/231554/set-beamercolorbox-height-automatically-to-sister-beamercolorbox-on-frametitle
\makeatletter
\setbeamertemplate{frametitle}{%
    \setbeamercolor{frametitle}{bg=doge, fg=white}
    \nointerlineskip%
    \usebeamerfont{headline}%
    \nointerlineskip%
    \hbox{\hspace{-0.09\paperwidth}%
    \begin{beamercolorbox}[wd=0.1\paperwidth,vmode]{secsubsec}%
        \newdimen\titleheight%
        \setbox0=\hbox{\usebeamerfont*{frametitle}\insertframetitle}
        \titleheight=\ht0 \advance\titleheight by \dp0%
        \vskip-.5pt%
        \vskip\titleheight%
        \ifx\insertframesubtitle\@empty%
          \strut\par%
        \else%
          \setbox0=\hbox{\usebeamerfont*{framesubtitle}\insertframesubtitle}%
          \titleheight=\ht0 \advance\titleheight by \dp0%
          \par{%
              \vskip\titleheight%
              \strut\par%
              \vskip-.65ex%
          }%
        \fi%
        \usebeamerfont{headline}%
        \vskip.5ex%
    \end{beamercolorbox}%
    \begin{beamercolorbox}[wd=0.99\paperwidth,leftskip=.3cm,rightskip=.3cm plus1fil,vmode]{frametitle}%
        \vskip.5ex%
        \usebeamerfont*{frametitle}\insertframetitle%
        \ifx\insertframesubtitle\@empty%
          \strut\par%
        \else%
          \par{\usebeamerfont*{framesubtitle}{\usebeamercolor[fg]{framesubtitle}\insertframesubtitle}\strut\par}%
        \fi%%
        \usebeamerfont{headline}%
        \vskip.5ex%
    \end{beamercolorbox}%
    }
    \nointerlineskip
}

% footer
\makeatother
\setbeamertemplate{footline}
{
  \leavevmode%
  \hbox{%
  \begin{beamercolorbox}[wd=.3\paperwidth,ht=2.25ex,dp=1ex,center]{author in head/foot}%
    \usebeamerfont{author in head/foot}\insertshortauthor
  \end{beamercolorbox}%
  \begin{beamercolorbox}[wd=.4\paperwidth,ht=2.25ex,dp=1ex,center]{title in head/foot}%
    \usebeamerfont{title in head/foot}\insertshorttitle
  \end{beamercolorbox}}%
  \begin{beamercolorbox}[wd=.3\paperwidth,ht=2.25ex,dp=1ex,center]{pagenum in head/foot}%
    \insertframenumber{} / \inserttotalframenumber\hspace*{1ex}
  \end{beamercolorbox}%
  \vskip0pt%
}
\makeatletter
\setbeamertemplate{navigation symbols}{}

\setbeamertemplate{section page}
{
    \begin{centering}
    \begin{beamercolorbox}[sep=12pt,center]{part title}
    \usebeamerfont{section title}\insertsection\par
    \end{beamercolorbox}
    \end{centering}
}

\AtBeginSection[]
{
    \setbeamertemplate{navigation symbols}{}
    \frame[plain,c,noframenumbering]{
        \sectionpage
        \tableofcontents[currentsection,subsectionstyle=hide]}
    \setbeamertemplate{navigation symbols}{\normalsize}
}